\chapter{Обучение классификатора}

\section{Выбор модели}

Заметим, что пространство признаков имеет очень большую размерность. К тому же многие из признаков явно взаимосвязаны с другими (действительно, доля досмотревших до 90 \% видео не может быть выше доли досмотревших до 50 \%). Если с первой из этих проблем можно справиться, например, воспользовавшись методом главных компонент \footcite{Pearson1901}, то избавиться от второй гораздо сложнее.

Так как борьба с вышеозначенными проблемами сама по себе является нетривиальной задачей, было принято решение использовать модели машинного обучения, основанные на решающих деревьях \footcite{LeoConsultant1984}. Это позволяет свести предобработку данных к минимуму. Однако решающие деревья в чистом виде редко используются из-за сложностей связанных с соблюдением баланса между сложностью модели и ее обобщающей способностью. Поэтому в качестве базовой модели был выбран случайный лес \footcite{Ho}. В качестве дополнительных моделей были выбраны чрезвычайно случайные деревья (extremely randomized trees) \footcite{Geurts2006}, также метод градиентного бустинга (gradient boosting) \footcite{Friedman2001} основанный на решающих деревьях. Первые два алгоритма были взяты из библиотеки scikit-learn \footcite{scikit-learn}, последний был взят из библиотеки CatBoost \footcite{Prokhorenkova2017}.

\section{Обучение модели}

Для подбора гиперпараметров и получения оценки точности и полноты использовался сеточный поиск с оценкой точности и полноты с использованием 5-кратной кроссвалидации. Для случайного леса и чрезвычайно случайных деревьев подбиралось количество деревьев в ансамбле. Для классификатора CatBoost кроме этого подбиралась максимальная глубина дерева.

Результаты сеточного поиска представлены в таблице \ref{tab:toloka_grid_search}, а также на рисунках \ref{fig:toloka_randomforest}, \ref{fig:toloka_extratrees} и \ref{fig:toloka_catboost}.

\begin{table}[h]
    \centering
    \begin{tabular}{|c|c|c|}
        \hline
        Модель & Наилучшая точность, \% & Наилучшая полнота, \% \\
        \hline
        RandomForestClassifier & 72.0 & 50.8 \\
        \hline
        ExtraTreesClassifier & 74.3 & 47.2 \\
        \hline
        CatBoostClassifier & 80.3 & 53.3 \\
        \hline
    \end{tabular}
    \caption{Результаты сеточного поиска}
    \label{tab:toloka_grid_search}
\end{table}

\begin{figure}
    \centering
    \includegraphics{./images/toloka_randomforest_precision.pdf}
    \includegraphics{./images/toloka_randomforest_recall.pdf}
    \caption{Результаты сеточного поиска для модели RandomForestClassifier}
    \label{fig:toloka_randomforest}
\end{figure}

\begin{figure}
    \centering
    \includegraphics{./images/toloka_extratrees_precision.pdf}
    \includegraphics{./images/toloka_extratrees_recall.pdf}
    \caption{Результаты сеточного поиска для модели ExtraTreesClassifier}
    \label{fig:toloka_extratrees}
\end{figure}

\begin{figure}
    \centering
    \includegraphics{./images/toloka_catboost_precision.pdf}
    \includegraphics{./images/toloka_catboost_recall.pdf}
    \caption{Результаты сеточного поиска для модели CatBoostClassifier}
    \label{fig:toloka_catboost}
\end{figure}

Наилучшей точности удалось достигнуть с использованием классификатора из библиотеки CatBoost (свыше 80.3 \%). Полнота у данного классификатора получилась выше, чем у конкурентов (около 51.5 \%).

При дальнейшем изучении предсказаний данной модели выяснились следующие факты: выборка содержит некоторое количество видео, неверно размеченных, как работающие, и плеерные события имеют большой вес в предсказании модели. Первая проблема требует другого способа разметки данных, и ее решение будет предложено далее. Для решения же второй проблемы было предложено добавить в признаки идентификатор плеера. Это позволило бы модели корректировать ожидаемый профиль событий в зависимости от конкретного плеера, который их шлет. Результаты этого эксперимента представлены на рисунке \ref{fig:toloka_catboost_with_cat}. К сожалению, точность классификатора упала до 72.6 \%, однако полнота классификации возросла почти на 17 \% и составила почти 68.2 \%.

\begin{figure}
    \centering
    \includegraphics{./images/toloka_catboost_with_cat_precision.pdf}
    \includegraphics{./images/toloka_catboost_with_cat_recall.pdf}
    \caption{Результаты сеточного поиска для модели CatBoostClassifier с  категориальным признаком "идентификатор плеера"}
    \label{fig:toloka_catboost_with_cat}
\end{figure}

\section{Одноплеерная модель}

Для решения проблемы неаккуратной разметки данных, было принято решение размечать выборку с помощью данных обхода. В таком случае возникает обратная проблема: у нас есть информация о том, какие видео не работали в выбранный день, однако у нас нет достоверного обратного сигнала. Для решения данной проблемы был выбран следующий подход: известно, что среди видеозаписей, которые представлены на выдаче поиска, примерно 1 \% не работает. Исходя из этого, при составлении выборки мы считаем количество достоверно неработавших видео исходя из логов обхода и им присваивается метка класса 1. Остальные видео ранжируются по среднему времени просмотра и выбирается в 99 раз большее количество видео, чем получившийся объем первого класса. Эти видео размечаются меткой класса 0 и также добавляются в выборку. Таким образом мы получаем размеченную выборку большого размера, так как нам доступна история обхода за довольно длительный период времени. Поэтому было решено учить модель только на данных одного плеера, что позволяет модели подстраиваться под конкретный профиль событий и тип контента. В качестве модели здесь выбран только CatBoostClassifier в силу того, что он зарекомендовал себя, а также должен лучше себя вести на сильно возросшей выборке. Результаты сеточного поиска в данном случае представлены на рисунке \ref{fig:model_factory}.

\begin{figure}
    \centering
    \includegraphics{./images/model_factory_precision.pdf}
    \includegraphics{./images/model_factory_recall.pdf}
    \caption{Результаты сеточного поиска для модели CatBoostClassifier обученной на данных обхода}
    \label{fig:model_factory}
\end{figure}

Мы видим заметный прирост точности классификации (более 9 \%) при практически неизменной полноте по сравнению с классификацией данных, размеченных с помощью сервиса Yandex Toloka. В результате удалось обучить классификатор из библиотеки CatBoost, который достиг точности классификации 89.3 \% и полноты 54.7 \%.