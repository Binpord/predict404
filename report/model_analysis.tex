\chapter{Анализ результатов}

Наилучшей точности предсказания в 89.3 \% удалось достигнуть с помощью одноплеерной модели. При этом такая точность уже позволяет говорить о том, что предсказание модели является достаточным критерием для прекращения показа видеозаписи. При этом модель обладает достаточной полнотой в 51.4 \%. Таким образом данную модель можно использовать в качестве дополнения для уже существующего решения в виде обхода видеозаписей. К сожалению, невысокая полнота все же не позволяет полностью заменить существующий механизм без ущерба для качества выдачи. Также данный подход обладает рядом недостатков. Во-первых, данный метод требует дополнительного механизма создания дополнительных моделей при добавлении на выдачу новых плееров. Во-вторых, Также данный подход не применим для плееров, у которых не доступны события плеера, так как исключение их из признаков ухудшает качество предсказания. Тем не менее для всех остальных плееров данный метод способен улучшить существующий механизм блокирования видеозаписей. Так, например, можно блокировать показ видеозаписей, предсказанных как неработающие, а для остальных применять уже существующий подход.

С другой стороны, при классифицировании видеозаписей всех плееров удалось достигнуть лишь более скромной точности в 80.3 \%, что не позволяет применять данный механизм из-за того, что в таком случае из выдачи пропадет достаточно большое количество работающих видеозаписей.

В дальнейшем планируется внедрение одноплеерной модели в сервис Yandex Video. Анализ и последовательное улучшение точности и полноты его предсказания. А также изучение других возможностей предсказывать неработающие видеозаписи по статистике просмотров и, в частности, по истории времени просмотра, например, с помощью техник, более специфичных для анализа временных рядов.