% Четкое и краткое обоснование выбора темы, обоснование ее актуальности, определение объекта и предмета исследования, целей и задач, перечень методов исследования; краткая формулировка научно-теоретической и практической значимости исследования; сведения об апробации результатов исследования (публикации, выступления на конференциях итд).

% Объект - неработающие видео / не открывающиеся страницы интернета / soft404. Обзор других работ можно взять из detecting soft404 pages.

% Предмет исследования - детектирование неработающих видео по статистике просмотров.

% Не забыть рассказать о том, что видео не возвращают 404 ERROR и вставить скриншот с неработающим видео. Но это, наверное, лучше уже в основной части.

% Еще сюда же про видеопоиск (с фоткой) и про iframe'ы.

\chapter{Введение}

Выбор темы обусловлен спецификой работы сервиса Видео Поиск Yandex, в команде разрабатывающей который я делал НИР. Сервис позволяет пользователю искать видео содержащиеся на различных ресурсах интернета. В силу большого количества контента, собранного с различных интернет ресурсов, хранить все видеозаписи не представляется возможным. Поэтому контент на выдаче предоставляется в виде iframe'ов. Это решает проблему хранения видеозаписей, однако порождает другую проблему, а именно необходимость контролировать работоспособность провайдеров контента. С этой целью по кликам пользователей собирается анонимная статистика: когда пользователь кликает по видеозаписи, включается таймер, когда же пользователь переключает видео, к нам приходит время просмотра предыдущей. Основной проблемой здесь является высокая степень загрязненности данных. Если во время просмотра или переключения видео у пользователя пропадет интернет соединение, нам может либо не прийти клик, либо прийти клик с невалидными данными. Также возможен случай, когда пользователь открывает видео и отходит от компьютера. В это время таймер продолжает считать время просмотра, несмотря на то, что видео может так и не загрузиться, например, в силу ошибки на сервере провайдера.

Задачей данной работы ставится создание модели, которая смогла бы предсказывать неработающие видео с высокой точностью, так чтобы видео можно было убирать из выдачи только на основании ее предсказания. В качестве метода исследования был выбран экспериментальный подход: сбор и разметка выборки, вычисление метрик точности и полноты классификатора с помощью кроссвалидации, обучение модели и применение ее на свежих данных с последующей разметкой результатов.