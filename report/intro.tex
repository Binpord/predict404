% Четкое и краткое обоснование выбора темы, обоснование ее актуальности, определение объекта и предмета исследования, целей и задач, перечень методов исследования; краткая формулировка научно-теоретической и практической значимости исследования; сведения об апробации результатов исследования (публикации, выступления на конференциях итд).

% Объект - неработающие видео / не открывающиеся страницы интернета / soft404. Обзор других работ можно взять из detecting soft404 pages.

% Предмет исследования - детектирование неработающих видео по статистике просмотров.

% Не забыть рассказать о том, что видео не возвращают 404 ERROR и вставить скриншот с неработающим видео. Но это, наверное, лучше уже в основной части.

% Еще сюда же про видеопоиск (с фоткой) и про iframe'ы.

\chapter{Введение}

Выбор темы обусловлен спецификой работы сервиса Yandex Video. В силу того, что контент данного сервиса выдается пользователю в виде плееров, интегрированных в сайт с помощью тэга iframe, у данного сервиса возникает необходимость вовремя детектировать неработающие видеозаписи и убирать их из выдачи. На данный момент эта задача решается с помощью периодического обхода видеозаписей с помощью специального механизма, умеющего эмулировать работу клиентского web браузера, а также умеющего нажимать на кнопку начала проигрывания видеозаписи и оценивать, началось ли воспроизведение. Данный механизм позволяет получать достаточно достоверный сигнал о недоступности видео, однако обладает рядом существенных минусов. К ним относится, например, невозможность проверить доступность видео в странах кроме России (так как все запросы территориально отправляются из России), то есть учесть специфику локальных блокировок сайтов и контента. Также существенным недостатком такого метода является сложность масштабирования. При увеличении числа видеозаписей требуется увеличивать скорость работы обхода, а это невозможно без установки дополнительного оборудования. Для устранения вышеперечисленных недостатков было предложено использовать статистику просмотров для детектирования неработающих документов.

С этой целью по сессиям пользователей собирается анонимная статистика: когда пользователь начинает просмотр видеозаписи, включается таймер, когда же пользователь переключает видео или закрывает страницу сервиса, время просмотра записывается для дальнейшего анализа. Вместе со временем просмотра сохраняются и события плеера, если они доступны. Основной проблемой таких данных является высокая степень загрязненности. Если во время просмотра или переключения видео у пользователя пропадет интернет соединение, запись о просмотре может не прийти или содержать неверные данные. Также возможен случай, когда пользователь включает проигрывание видеозаписи, а затем отходит от компьютера. В это время таймер считает время просмотра, несмотря на то, что видео могло оказаться неработающим и так и не загрузиться.

Задачей данной работы ставится изучение возможности создания модели, которая смогла бы предсказывать неработающие видеозаписи по статистике просмотров с высокой точностью и приемлемой полнотой. В качестве метода исследования был выбран экспериментальный подход: сбор и разметка выборки, оценка точности и полноты классификации с помощью метода перекрестной проверки и анализ предсказаний полученной модели.