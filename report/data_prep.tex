\chapter{Подготовка данных}

\section{Исходные данные}

В качестве исходных данных имеется таблица, каждая строка которой содержит некоторый идентификатор плеера, идентификатор видеозаписи, а также время просмотра и события плеера, если они доступны. Также из разметки сайта провайдера контента берется продолжительность видеозаписи. Пример исходных данных представлен в таблице \ref{tab:raw_data}.

\begin{table}[h]
    \centering
    \begin{tabular}{|c|c|c|c|c|c|}
        \hline
        Плеер & Видео & Длительность, с & Просмотр, с & Старт & Ошибка \\
        \hline
        Плеер1 & Видео1 & 100 & 10 & True & \\
        \hline
        Плеер2 & Видео2 & 200 & 180 & True & \\
        \hline
        Плеер3 & Видео3 & 0 & 5 & & True \\
        \hline
        \ldots & \ldots & \ldots & \ldots & \ldots & \ldots \\
        \hline
    \end{tabular}
    \caption{Пример исходных данных.}
    \label{tab:raw_data}
\end{table}

\section{Выбор признаков}

Так как данные сильно зашумлены, клики необходимо усреднять. При этом события плеера~--- достаточно надежный источник информации, тем не менее они имеют несколько существенных недостатков. К этим недостаткам относится тот факт, что события плеера доступны не для всех плееров, а также каждому отдельному плееру характерен некоторый профиль событий для различных состояний видеозаписи, однако эти профили могут существенно отличаться для двух разных плееров. Существуют два различных способа борьбы с данными недостатками: во-первых можно сделать больший упор на время просмотра и производимые из него признаки, как на более универсальный источник информации, во-вторых можно попытаться сделать отдельную модель для каждого плеера, чтобы она смогла в большей мере использовать сигнал событий плеера, если они доступны. В ходе данной работы были опробованы оба подхода.

Также в силу высокого уровня зашумленности исходных данных понятно, что их необходимо каким-то образом усреднять. В случае с данным событий видеоплеера усреднение очевидно~--- если событие произошло, то данной записи ставится в соответствие единица, в противном случае ноль и в качестве признака используется среднее значение, то есть доля просмотров, в которых данное событие наблюдалось.

Из времени просмотра было выбрано сгенерировано большее количество признаков исходя из некоторых предположений о природе данных. Первое предположение заключается в том, что, если видеозапись не работает, большинство пользователей заметят это и переключать видео за довольно короткий промежуток времени, близкий к 30-40 секундам. Поэтому в качестве первой группы признаков были выбраны доля пользователей смотревших видео не менее 15, 30 и 45 секунд. Следующая идея говорит о том, что информация о том, как хорошо пользователи смотрят данное видео хранится в распределении доли просмотра контента. Для простоты и однообразности расчета распределение было введено в выборку в виде долей пользователей, досмотревших видео до 5, 10, 20, 30, 40, 50, 60, 70, 80 и 90 \% длительности видеозаписи. Последнее наблюдение говорит о том, что если видеозапись попала на выдачу, то когда-то она была обнаружена обходом, а значит в какой-то момент времени она работала. Таким образом чаще всего мы занимаемся не поиском неработающего контента, а пытаемся поймать момент, в который видеозапись перестала работать. В таких случаях принято говорить о данных в форме временного ряда, то есть последовательных значениях, зависящих от времени. Однако для удобства работы с данными и расширения круга потенциальных алгоритмов классификации хотелось бы иметь данные в виде вектора фиксированной размерности. Также важно помнить о шуме в данных, а значит и необходимости усреднения данных. Для соблюдения всех ограничений было принято решение воспользоваться методом скользящего среднего. В таком случае мы получаем, например, такие признаки как среднее время просмотра за 5 последних просмотров заканчивая последним записанным просмотром и такое же среднее за 5 просмотров заканчивая предпоследним просмотром и так далее. Хотелось бы заметить, что подход со скользящим средним применим не только к времени просмотра. Скользящее среднее также считалось для событий плеера и долей смотревших видео хотя бы 15, 30, 45 секунд и так далее.

В итоге для каждого видео мы получили вектор признаков, размерность которого составила порядка 150. После того как были выбраны признаки и собрана выборка, разметка видеозаписей производилась с помощью сервиса Yandex Toloka. Таким образом был собран датасет из порядка 100000 размеченных объектов.