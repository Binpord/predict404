\chapter{Подготовка данных}

\section{Сырые данные}

В качестве сырых данных имеется таблица, каждая строка которой содержит некоторый идентификатор плеера, идентификатор видео, а также время просмотра и плеерные события, если они доступны для данного плеера. Также из разметки сайта провайдера контента берется продолжительность видеозаписи. Здесь и далее строку в данной таблице я буду называть кликом.

Время просмотра вычисляется, как время между переключением видеозаписи или закрыванием страницы. Также здесь учитывается возможность того, что пользователь поставит видео на паузу. Плеерные события~--- это часть публичного API многих видеохостингов. Если провайдер контента позаботился об удобстве разработчиков, которые будут эмбедить их контент, они предоставляют возможность "подписаться" на их события, такие как смена разрешения видеозаписи, выставление плеера на паузу, ошибка воспроизведения и прочие. Такую функциональности предоставляет, например, сервис YouTube \footcite{YouTubeAPI}.

Разметка данных производилась при помощи сервиса Yandex Toloka \footcite{Toloka}.

\section{Выбор признаков}

Так как данные сильно зашумлены, клики необходимо усреднять. При этом плеерные события~--- достаточно полезный и надежный источник информации, однако они доступны не для всех плееров, поэтому хотелось бы сделать больший упор именно на время просмотра, как на более универсальный источник информации. В качестве признаков для модели были выбраны доля пользователей, которые смотрели видео в течении не менее 15, 30 и 45 секунд, а также доля досмотревших до 5, 10, 20, 30, 40, 50, 60, 70, 80 и 90 \% видео. Также в качестве признаков использовались усредненные события старта и ошибки (если событие имело место, то клику ставилась единица, в случае отсутствия события 0 и усреднялось по доступным кликам). Кроме этого было замечено, что фактически мы имеем дело с временным рядом времени просмотра от момента времени и пытаемся "поймать" момент, когда плеер перестал играть видеозапись. В таком случае полезными показались такие признаки как скользящее среднее времени просмотра и событий старта и ошибки.