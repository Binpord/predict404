% Последовательное, логически стройное изложение итогов исследования в соответствии с целью и задачами, поставленными и сформулированными во введении. Заключение может включать в себя практические предложения, что повышает ценность теоретического материала, но не должно повторять введение.

\chapter{Заключение}

Данная работа была посвящена изучению частного случая проблемы soft 404, а именно детектированию неработающих видеозаписей. Так как ни один из ранее предложенных способов решения данной проблемы не подходит в данном случае из-за специфики задачи, было предложено использовать статистику просмотров для детектирования неработающих видеозаписей. Благодаря такому подходу удалось обучить классификатор из библиотеки CatBoost, который показал точность классификации 80.3 \% и полноту 51.4 \%. Также был предложен способ ограничиться одним плеером, что позволило увеличить точность классификации на 9 процентных пунктов при неизменной полноте. Таким образом был построен классификатор, обладающий точностью предсказания 89.3 \% и полнотой 51.4 \%.