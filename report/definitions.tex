\chapter{Используемые определения}

\section{Видеоплеер}

Под видеоплеером (плеером) в данной работе подразумевается элемент web страницы, позволяющий пользователю просматривать видеозаписи (смотри рисунок \ref{fig:youtube_good}). Как правило, один видеоплеер умеет воспроизводить ряд видеозаписей, хранящихся на сервере провайдера контента.

Помимо предоставления конечным пользователям возможности просматривать видеозаписи, видеохостинги часто предоставляют возможность другим интернет ресурсам интегрировать этот контент в свои страницы. Для этого, как правило, используется тэг iframe языка разметки HTML. Данный тэг, согласно спецификации \cite{HTML53}, позволяет web странице открывать внутри себя вложенные страницы, например, с видеоплеером.

Кроме того ряд крупных видеохостингов предоставляют более обширное API для интеграции контента. В частности, некоторые плееры открывают доступ к так называемым событиям плеера, то есть оповещениям о некоторых событиях, которые дают возможность оценивать состояние контента. К таким событиям чаще всего относятся события старта, паузы и ошибки плеера. Примером плеера, который предоставляет подобную функциональность может служить плеер видеохостинга YouTube \cite{YouTubeAPI}.

\begin{figure}
    \centering
    \includegraphics[width=\textwidth]{../images/youtube_good.png}
    \caption{Видеоплеер YouTube.}
    \label{fig:youtube_good}
\end{figure}

\section{Неработающие видео}

Под неработающим видео (рисунок \ref{fig:bad_video}) в данной работе поднимается любая видеозапись, которую пользователь не может просмотреть. Видеозапись может прекратить работу по ряду причин: отключение сервера видеохостинга, на котором данная видеозапись хранилась или блокировка просмотра видеозаписи на основании обращения правообладателя контента.

\begin{figure}
    \centering
    \includegraphics[width=\textwidth]{../images/youtube_bad.png}
    \includegraphics[width=\textwidth]{../images/yandex_video_bad.png}
    \caption{Пример неработающего видео на сайте www.youtube.com и на сайте Yandex Video.}
    \label{fig:bad_video}
\end{figure}

\section{Бинарная классификация}

В данной работе рассматривается задача бинарной классификации, то есть разбиения множества объектов на два класса. Были выбраны следующие обозначения: под классом 0 (нулевым классом, отрицательным классом) подразумевается класс работающих видео, а под классом 1 (положительным классом) подразумевается класс неработающих видео, детектирование которых и является целью данной работы.

При анализе предсказания модели в задаче бинарной классификации принято говорить о четырех группах объектов: ложно отрицательные объекты (false negatives, $FN$), ложно положительные объекты (false positives, $FP$), верно отрицательные объекты (true negatives, $TN$) и верно положительные объекты (true positives, $TP$).

Для оценки модели в задаче бинарной классификации принято использовать метрики точности (precision, $P$) и полноты (recall, $R$), которые вводятся следующим образом:
\[
    P = \frac{TP}{TP + FP},\ R = \frac{TP}{TP + FN}.
\]
Здесь и далее в данной работе под точностью и полнотой классификации подразумеваются именно введенные выше метрики.

\section{Решающие деревья}

Решающие деревья \cite{Breiman2017}~--- это модель машинного обучения, которая позволяет решать задачи классификации и регрессии с помощью построения двоичного дерева, подобного дереву поиска. В каждом внутреннем узле данного дерева находится условие, исходя из соответствия которому для конкретного объекта поиск спускается либо в правое, либо в левое поддерево. В листьях данного дерева находится предсказание модели, то есть метка класса в случае задачи классификации либо численное значение в случае задачи регрессии.

Решающие деревья являются одной из самых интерпретируемых моделей машинного обучения, так как финальная модель может быть очевидным способом разложена в ряд логических предикатов и соответствующих им значений предсказания. Кроме того данная модель является устойчивой к высокой размерности данных и зависимости среди признаков, что позволяет сводить предобработку данных к минимуму при работе с данной моделью. Тем не менее данная модель имеет и ряд существенных минусов. Так предсказание решающего дерева очень чувствительно к гиперпараметру глубины дерева. Действительно, если глубина не ограничена сверху, то модель легко переобучается (производя деление до тех пор, пока в каждом листе не останется единственный объект обучающей выборки). Предсказания такой модели скорее всего выйдут очень шумными, а обобщающая способность модели окажется крайне низкой. В то же время слишком жесткое ограничение на глубину деления вызовет создание модели, сложности которой недостаточно для описания зависимостей в данных. В то же время, даже обладая априорным знанием оптимальной глубины дерева, нахождение оптимального дерева остается нетривиальной задачей, которой посвящен ряд статей (например \cite{Masa}). Все эти проблемы вместе с развитием других моделей машинного обучения привели к тому что на данный момент решающие деревья в чистом виде практически не используются, однако они являются важным составляющим компонентом для более сложных и эффективных моделей.

\section{Bootstrap aggregation}

Bootstrap aggregation (bagging) \cite{breiman1996bagging}~--- это метод, применяемый в моделях машинного обучения для улучшения стабильности и точности предсказания. Метод заключается в следующем: пусть у нас есть выборка из $n$ объектов, давайте сгенерируем из нее $m$ подвыборок размера $n'$ путем последовательного выбора произвольного объекта исходной выборки. Давайте теперь обучим на полученных выборках модели и усредним их предсказания. Можно доказать, что в случае независимости предсказаний полученных моделей, дисперсия результирующего предсказания падает с увеличением их числа.

Изначально метод никак не связан с решающими деревьями, однако на данный момент чаще всего bagging применяется именно для моделей решающего дерева. Это связано с тем, что для таких моделей проще обеспечить независимость предсказания отдельных моделей путем дополнительной подвыборки признаков, по которым разрешено ветвиться каждому конкретному дереву в ансамбле, а также довольно мягкому ограничению глубины деревьев (предполагается, что переобученные на разных подвыборках деревья будут давать существенно независимые предсказания). В таком случае также не сильно существенно строить оптимальное дерево каждый раз. Поэтому чаще всего выбор признака и значения для дальнейшего ветвления на каждом шагу производится жадным образом, что позволяет значительно ускорить время обучения модели. На этом основана модель random forest \cite{Ho}, которая использовалась в данной работе, как базовая модель. Реализация модели была взята из библиотеки Scikit-learn \cite{scikit-learn}. С ее помощью удалось достигнуть точности предсказания неработающих видео в 72.0 \% и полноты предсказания в 50.8 \%.

\section{Модель extremely randomized trees}

Модель extremely randomized trees \cite{Geurts2006}~--- это модель, во многом похожая на модель random forest. Для ее построения также производится генерация подвыборок в соответствии с методом bagging и обучение на подвыборках моделей решающего дерева. Однако в отличии от жадного поиска оптимального признака и значения для ветвления при обучении каждого дерева в ансамбле, как это происходит при обучении модели random forest, модель extremely randomized trees произвольно генерирует несколько предполагаемых вариантов и выбирает оптимальный среди них. Это позволяет сделать деревья в ансамбле еще более независимыми, что в ряде случаев позволяет улучшить точность предсказания. Реализация данной модели также была взята из библиотеки Scikit-learn \cite{scikit-learn}. Использование позволило увеличить точность предсказания до 74.3 \%, однако при этом полнота упала до 47.0 \%.

\section{Gradient boosting}

Gradient boosting (boosting) \cite{Friedman2001}~--- это метод машинного обучения, позволяющий создавать ансамбль из простых моделей, каждая из которых обладает достаточно низкой предсказательной способностью, обладающий высокой точностью предсказания. Этого удается достичь путем последовательного обучения моделей: первая модель обучается на исходных данных, каждая последующая же модель учится предсказывать уже не искомое значение, а ошибку получившегося до нее ансамбля (в случае задачи регрессии, в случае задачи классификации она учится предсказывать класс объекта, но больший вес отдается тем объектам, на которых предыдущий ансамбль ошибается). В соответствии с точностью ее предсказания, получившейся модели присваивается вес и она пополняет ансамбль.

Несмотря на то, что данный метод также не имеет непосредственного отношения к решающим деревьям, чаще всего в качестве базовых моделей в ансамбле выбираются именно решающие деревья небольшой глубины. Реализация данной модели была взята из библиотеки CatBoost \cite{Prokhorenkova2017}. Модель показала наилучшую точность из полученных в 80.3 \%. Полнота получившейся модели оказалась также выше чем у моделей random forest и extremely randomized trees и составила 51.4 \%. Также для данной модели был произведен эксперимент с предсказанием неработающих видеозаписей для одного плеера. В данном эксперименте удалось достигнуть точности в 89.3 \% при той же полноте.

\section{Yandex Video}

Yandex Video \cite{VideoSearch}~--- это сервис, позволяющий пользователям смотреть видеозаписи, собранные с множества различных видеохостингов. Из-за большого объема контента и невозможности моментально реагировать на его изменения, хранить видео на серверах компании Yandex не представляется возможным. Поэтому видеозаписи показываются пользователям в виде интегрированных с помощью тэга iframe видеоплееров.

\section{Yandex Toloka}

Yandex Toloka \cite{Toloka}~--- это сервис, позволяющий публиковать некоторые несложные задания, которые другие пользователи могут выполнять за материальное вознаграждение. Данный сервис использовался в работе для получения экспертной разметки данных: пользователям, взявшимся за выполнение задания, показывался iframe с видеозаписью и требовалось ответить, проигрывается эта видеозапись или нет. Для увеличения точности разметки, одна и та же видеозапись размечалась несколько раз разными пользователями. На разметку отправлялись видеозаписи, для которых за предыдущий день набралось хотя бы 100 зафиксированных просмотров. Таким образом за два раза была размечена выборка размером в 101034 видеозаписи. Неработающие видеозаписи составили 1.36 \% выборки.