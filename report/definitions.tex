\chapter{Используемые определения}

\section{Бинарная классификация}

В данной работе рассматривается задача бинарной классификации, то есть разбиения множества объектов на два класса. Были выбраны следующие обозначения: под классом 0 (нулевым классом, отрицательным классом) подразумевается класс работающих видео, а под классом 1 (положительным классом) подразумевается класс неработающих видео, детектирование которых и является целью данной работы.

При анализе предсказания модели в задаче бинарной классификации принято говорить о четырех группах объектов: объекты, верно отнесенные к отрицательному классу (true negatives); объекты, верно отнесенные к положительному классу (true positives); объекты, неверно отнесенные к отрицательному классу (false negatives) и объекты, неверно отнесенные к положительному классу (false positives).

Для оценки модели в задаче бинарной классификации принято использовать метрики точности (precision) и полноты (recall), которые вводятся следующим образом:
\[
    P = \frac{TP}{TP + FP},\ R = \frac{TP}{TP + FN},
\]
где с помощью $P$ обозначена точность (precision), $R$~--- полнота (recall), $TP$, $FP$ и $FN$~--- числа объектов, относящихся к той или иной группе из четырех указанных выше. Так, $TP$~--- это число объектов, верно отнесенных к положительному классу (true positives), $FP$~--- это число объектов, неверно отнесенных к положительному классу (false positives), а $FN$~--- это число объектов, неверно отнесенных к отрицательному классу (false negatives). Здесь и далее под точностью и полнотой классификации подразумеваются именно введенные выше метрики.

\section{Видеоплеер}

Под видеоплеером (либо просто плеером) в данной работе подразумевается элемент web страницы, позволяющий пользователю смотреть видеозаписи. Один видеоплеер, как правило, умеет воспроизводить ряд видеозаписей, хранящихся на сервере провайдера контента. Различные видеохостинги могут использовать один общий плеер, однако внутри одного хостинга видео, как правило, воспроизводятся с помощью одного и того же плеера.

Помимо предоставления конечным пользователям возможности просматривать видеозаписи, видеохостинги часто предоставляют возможность другим интернет ресурсам интегрировать этот контент в свои страницы. Для этого, как правило, используется тэг iframe языка разметки HTML. Данный тэг, согласно спецификации \footcite{HTML53}, позволяет web странице открывать внутри себя вложенные страницы, например, с видеоплеером.

Кроме того ряд крупных видеохостингов предоставляют более обширное API для интеграции контента. В частности, некоторые плееры открывают доступ к так называемым событиям плеера, то есть оповещениям о некоторых событиях, которые дают возможность оценивать состояние контента. К таким событиям чаще всего относятся события старта, паузы и ошибки плеера. Примером плеера, который предоставляет подобную функциональность может служить плеер видеохостинга YouTube \footcite{YouTubeAPI}.

\section{Yandex Toloka}

Yandex Toloka (яндекс толока, толока) \footcite{Toloka}~--- это сервис, позволяющий публиковать некоторые несложные задания, которые другие пользователи могут выполнять за материальное вознаграждение. Данный сервис использовался в данной работе для получения экспертной разметки данных: пользователям, взявшимся за выполнение задания, показывался iframe с видеозаписью и требовалось ответить, проигрывается эта видеозапись или нет.