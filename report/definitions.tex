\chapter{Используемые определения}

\section{Бинарная классификация}

В данной работе рассматривается задача бинарной классификации, то есть разбиения множества объектов на два класса. Были выбраны следующие обозначения: под классом 0 (нулевым классом, отрицательным классом) подразумевается класс работающих видео, а под классом 1 (положительным классом) подразумевается класс неработающих видео, детектирование которых и является целью данной работы.

При анализе предсказания модели в задаче бинарной классификации принято говорить о четырех группах объектов: ложно отрицательные объекты (false negatives, $FN$), ложно положительные объекты (false positives, $FP$), верно отрицательные объекты (true negatives, $TN$) и верно положительные объекты (true positives, $TP$).

Для оценки модели в задаче бинарной классификации принято использовать метрики точности (precision, $P$) и полноты (recall, $R$), которые вводятся следующим образом:
\[
    P = \frac{TP}{TP + FP},\ R = \frac{TP}{TP + FN}.
\]
Здесь и далее в данной работе под точностью и полнотой классификации подразумеваются именно введенные выше метрики.

\section{Видеоплеер}

Под видеоплеером (плеером) в данной работе подразумевается элемент web страницы, позволяющий пользователю смотреть видеозаписи (рисунок \ref{fig:youtube_good}). Один видеоплеер, как правило, умеет воспроизводить ряд видеозаписей, хранящихся на сервере провайдера контента. Различные видеохостинги могут использовать один общий плеер, однако внутри одного хостинга видео, как правило, воспроизводятся с помощью одного и того же плеера.

Помимо предоставления конечным пользователям возможности просматривать видеозаписи, видеохостинги часто предоставляют возможность другим интернет ресурсам интегрировать этот контент в свои страницы. Для этого, как правило, используется тэг iframe языка разметки HTML. Данный тэг, согласно спецификации \cite{HTML53}, позволяет web странице открывать внутри себя вложенные страницы, например, с видеоплеером.

Кроме того ряд крупных видеохостингов предоставляют более обширное API для интеграции контента. В частности, некоторые плееры открывают доступ к так называемым событиям плеера, то есть оповещениям о некоторых событиях, которые дают возможность оценивать состояние контента. К таким событиям чаще всего относятся события старта, паузы и ошибки плеера. Примером плеера, который предоставляет подобную функциональность может служить плеер видеохостинга YouTube \cite{YouTubeAPI}.

\begin{figure}
    \centering
    \includegraphics[width=\textwidth]{../images/youtube_good.png}
    \caption{Видеоплеер YouTube.}
    \label{fig:youtube_good}
\end{figure}

\section{Yandex Video}

Yandex Video \cite{VideoSearch}~--- это сервис, позволяющий пользователям смотреть видеозаписи, собранные с множества различных видеохостингов (рисунок \ref{fig:yandex_video}). Из-за большого объема контента и невозможности моментально реагировать на его изменения, хранить видео на локальных серверах не представляется возможным, а потому видеозаписи показываются пользователям в виде интегрированных с помощью тэга iframe видеоплееров.

\begin{figure}
    \centering
    \includegraphics[width=\textwidth]{../images/yandex_video.png}
    \caption{Сервис Yandex Video.}
    \label{fig:yandex_video}
\end{figure}

\section{Yandex Toloka}

Yandex Toloka \cite{Toloka}~--- это сервис, позволяющий публиковать некоторые несложные задания, которые другие пользователи могут выполнять за материальное вознаграждение. Данный сервис использовался в данной работе для получения экспертной разметки данных: пользователям, взявшимся за выполнение задания, показывался iframe с видеозаписью и требовалось ответить, проигрывается эта видеозапись или нет.

\section{Неработающие видео}

Под неработающим видео (рисунок \ref{fig:bad_video}) в данной работе поднимается любая видеозапись, которую пользователь не может просмотреть. Видеозапись может прекратить работу по ряду причин: отключение сервера видеохостинга, на котором данная видеозапись хранилась или блокировка просмотра видеозаписи на основании обращения правообладателя контента.

\begin{figure}
    \centering
    \includegraphics[width=\textwidth]{../images/youtube_bad.png}
    \includegraphics[width=\textwidth]{../images/yandex_video_bad.png}
    \caption{Пример неработающего видео на сайте www.youtube.com и на сайте Yandex Video.}
    \label{fig:bad_video}
\end{figure}

\section{Решающие деревья}

Решающие деревья \cite{Breiman2017}~---