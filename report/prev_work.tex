\chapter{Анализ предметной области}

Наиболее близкой задачей к поставленной является задача детектирования "soft 404" страниц. В соответствии с протоколом HTTP \cite{Fielding2014}, если при обращении к серверу клиент запрашивает документ, который не доступен по той или оной причине, сервер должен возвращать ошибку. Как правило это ошибка 403 (Forbidden Error) или 404 (Not Found). Тем не менее многие интернет ресурсы стремятся предоставить пользователю более дружественный интерфейс и вместо возврата ошибки имитируют нормальную работу, возвращая код возврата 200 (ОК), а вместо запрашиваемого документа отображают страницу, оповещающую пользователя о причинах недоступности контента.

Подобное поведение делают задачу детектирования недоступных документов нетривиальной. Первая попытка борьбы с данной проблемой была предпринята в статье от 2004 года \cite{Bar-Yossef2004}. Авторы данной статьи предлагают спровоцировать сервер вернуть заведомо недоступный документ путем прибавления к имеющемуся названию искомого документа произвольного окончания. После этого предлагается сравнивать поведение сервера в случае обращения к исходному документу и к заведомо недоступному. Второе упоминание о проблеме soft 404 датируется 2012 годом \cite{Meneses2012}. В отличии от первой статьи, данная работа целиком посвящена проблеме детектирования soft 404 страниц. Авторы этой статьи предлагают использовать классификатор, который использует в качестве данных лексические сигнатуры, содержащиеся в заголовке или в тексте страницы.

К сожалению ни один из предложенных способов детектирования soft 404 документов не может быть применен к задаче детектирования неработающих видео, так как они оба предполагают периодический обход видеозаписей, а данный подход обладает рядом недостатков, о чем было сказано во введении. Кроме того важно понимать, что в отличии от других интернет ресурсов, видеохостинги при обращении к несуществующему или уже удаленному документу чаще всего предпочитают показывать страницу, полностью идентичную странице, которая была бы отображена в случае, если видео было доступно для просмотра. Сообщение о недоступности видеозаписи как правило располагается в самом видеоплеере. Таким образом подход, основанный на детектировании переадресаций на заранее заготовленные документы, содержащие оповещение о недоступности видео не применим так как в данном случае переадресаций не происходит. Анализ лексических сигнатур, содержащихся на странице и ее заголовке также неприменим в силу того, что и тот и другой текст не зависит от доступности видеозаписи, а само сообщение о возникшей проблеме находится в плеере. Таким образом для детектирования таких сообщений требуется технически сложный механизм, который умеет эмулировать доступ к странице из пользовательского web браузера и попытку воспроизведения видеозаписи, то есть данный подход не позволяет даже упростить существующее на данный момент решение.