\documentclass{beamer}
\usepackage[T1,T2A]{fontenc}
\usepackage[utf8]{inputenc}
\usepackage[english,russian]{babel}
\usepackage{tempora}   % Times New Roman font
\usepackage{newtxmath} %


\usetheme{Madrid}

\setbeamertemplate{footline}[page number]
\setbeamertemplate{navigation symbols}{}

\title{Детектирование неработающих видео \\
по статистике просмотров}
\author{Автор: Шиянов В.А. \\
Научный руководитель: к.ф.-м.н. Протасов С.И.}
\institute{МФТИ}
\date{\today}

\begin{document}

\begin{frame}
    \titlepage
\end{frame}

% \begin{frame}{Содержание}
%     \tableofcontents
% \end{frame}

\section{Введение}

\subsection{Актуальность}
\subsection{предмет исследования}

\begin{frame}{Soft 404 страницы}
    \includegraphics[width=\textwidth]{../images/github_soft404.png}
\end{frame}

\begin{frame}{Неработающие видео}
    \includegraphics[width=\textwidth]{../images/yandex_video_bad.png}
\end{frame}

\section{Предлагаемое решение}

\subsection{Данные}

\begin{frame}{Статистика просмотров}
    \begin{table}
        \centering
        \begin{tabular}{|c|c|c|c|c|c|}
            \hline
            Плеер & Видео & Длительность, с & Просмотр, с & Старт & Ошибка \\
            \hline
            Плеер1 & Видео1 & 100 & 10 & True & \\
            \hline
            Плеер2 & Видео2 & 200 & 180 & True & \\
            \hline
            Плеер3 & Видео3 & 0 & 5 & & True \\
            \hline
            \ldots & \ldots & \ldots & \ldots & \ldots & \ldots \\
            \hline
        \end{tabular}
    \end{table}
\end{frame}

\subsection{Обучение модели}

\begin{frame}{Обучение модели}
    \includegraphics[width=\textwidth]{../images/toloka_catboost_precision.pdf}
\end{frame}

% \begin{frame}{Обучение модели}
%     \includegraphics[width=\textwidth]{../images/toloka_catboost_recall.pdf}
% \end{frame}


\begin{frame}{Обучение модели}
    \begin{table}
        \centering
        \begin{tabular}{|c|c|c|}
            \hline
            Модель & Точность, \% & Полнота, \% \\
            \hline
            RandomForestClassifier, 500 деревьев & 72.0 & 50.8 \\
            \hline
            ExtraTreesClassifier, 1400 деревьев & 74.3 & 47.0 \\
            \hline
            ExtraTreesClassifier, 1100 деревьев & 73.9 & 47.2 \\
            \hline
            CatBoostClassifier, 650 деревьев глубины 3 & \textbf{80.3} & 51.4 \\
            \hline
            CatBoostClassifier, 500 деревьев глубины 4 & 77.5 & \textbf{53.2} \\
            \hline
        \end{tabular}
    \end{table}
\end{frame}

\begin{frame}{Анализ модели}
    События плеера имеют большой вес в результирующей модели. \\
    Попробуем добавить идентификатор плеера в признаки.
    \begin{table}
        \centering
        \begin{tabular}{|c|c|c|}
            \hline
            Модель & Точность, \% & Полнота, \% \\
            \hline
            CatBoostClassifier, 1250 деревьев глубины 2 & \textbf{72.6} & 65.0 \\
            \hline
            CatBoostClassifier, 800 деревьев глубины 5 & 71.3 & \textbf{68.1} \\
            \hline
        \end{tabular}
    \end{table}
\end{frame}

\begin{frame}{Одноплеерная модель}
    Попробуем обучить модель на данных единственного плеера. \\
    Для разметки воспользуемся историей обхода.
    \begin{table}
        \centering
        \begin{tabular}{|c|c|c|}
            \hline
            Модель & Точность, \% & Полнота, \% \\
            \hline
            CatBoostClassifier, 1550 деревьев глубины 3 & \textbf{89.3} & 51.4 \\
            \hline
            CatBoostClassifier, 1850 деревьев глубины 5 & 88.2 & \textbf{54.7} \\
            \hline
        \end{tabular}
    \end{table}
\end{frame}

\section{Заключение}

\begin{frame}{Дальнейшие планы}
    \begin{itemize}
        \item внедрение одноплеерной модели, как вспомогательного для обхода механизма, анализ и последовательное улучшение точности и полноты ее предсказания;
        \item изучение других возможностей предсказывать неработающие видеозаписи по статистике просмотров.
    \end{itemize}
\end{frame}

\begin{frame}{Заключение}
    \begin{itemize}
        \item была изучена предметная область;
        \item было предложено использовать статистику просмотров для детектирования неработающих видео;
        \item был предложен ряд моделей машинного обучения, удалось добиться точности в 80.3 \% и полноты в 51.4 \%;
        \item был предложен одноплеерный подход, с помощью которого удалось достичь точности в 89.3 \% и полноты в 51.4 \%.
    \end{itemize}
\end{frame}

\end{document}