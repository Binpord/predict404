% Посвящается раскрытию предмета исследования. Количество глав (разделов) и их характер зависят от особенностей темы, цели и задач, выбранных методов и т. п. Содержание глав (разделов) ВКР должно продемонстрировать умение автора логично, последовательно, сжато и аргументированно излагать материал, выявлять новое и оригинальное в ходе разработки рассматриваемой проблемы.

% iframe
% soft404

\chapter{Предыдущие работы}

Наиболее близкой задачей к поставленной является задача детектирования так называемого мягкого 404 (мягкой ошибки 404, soft 404). В соответствии с протоколом HTTP \footcite{Fielding2014}, когда при обращении к серверу клиент запрашивает документ, который более не доступен, сервер должен возвращать ошибку. Как правило это 404 (Not Found) код ошибки. Мягкий 404 чаще всего возникает в следствие попыток сайта предоставить пользователю более дружественный интерфейс. В таком случае, когда пользователь запрашивает недоступный документ, сервер возвращает не ошибку 404, а код 200 (ОК) и страницу, на которой в более удобочитаемом формате написано, что документ недоступен.

В таком случае детектирование недоступных документов становится нетривиальной задачей. Мне удалось найти две статьи, авторы которых подходят к ее решению. В первой из них \footcite{Bar-Yossef2004} предлагается "спровоцировать" сервер вернуть недоступный документ и сравнивать поведение сервера в случае обращения к искомому документу и к заведомо недоступному. Во второй статье \footcite{Meneses2012} авторы предлагают классифицировать страницы используя лексические сигнатуры, содержащиеся в заголовке страницы.

Схожесть задачи детектирования неработающих видео с задачей детектирования мягкого 404 обусловлена тем, что видеохостинги также нередко предпочитают информировать пользователя о недоступности видео с помощью сообщения в плеере (смотри рисунок \ref{fig:bad_video}). Особенно часто это проявляется в случае с видеозаписями, которые были доступны ранее, а затем стали недоступными для просмотра по той или иной причине.

\begin{figure}
    \centering
    \includegraphics[width=\textwidth]{images/youtube_bad}
    \includegraphics[width=\textwidth]{images/yandex_video_bad}
    \caption{Пример неработающего видео на сайте www.youtube.com и на сайте видеопоиска.}
    \label{fig:bad_video}
\end{figure}

К сожалению ни один из способов детектирования мягкого 404 не может быть применен к задаче детектирования неработающих видео. Это вызвано тем, что в данном случае текст почти всегда показывает человеку плеер, который работает внутри веб браузера. Таким образом простого скачивания html файла уже недостаточно. Требуется полноценный механизм, способный эмулировать нажатие на кнопку проигрывания и оценивать началось ли воспроизведение видео. Метод же, предлагаемый в данной работе, базируется на использовании статистики просмотров видеозаписи, что позволяет избежать использования технически сложных средств.