% Посвящается раскрытию предмета исследования. Количество глав (разделов) и их характер зависят от особенностей темы, цели и задач, выбранных методов и т. п. Содержание глав (разделов) ВКР должно продемонстрировать умение автора логично, последовательно, сжато и аргументированно излагать материал, выявлять новое и оригинальное в ходе разработки рассматриваемой проблемы.

% iframe
% soft404

\chapter{Предыдущие работы}

Задача детектирования неработающих видео во многом схожа с задачей детектирования soft 404 страниц интернета, рассматриваемую в работе \footcite{Meneses2012}. Это сходство обосновывается тем, что неработающие видео также не возвращают HTTP ERROR 404 Not Found. Вместо этого, как правило, плееры вместо видео отображают текст (смотри рисунки \ref{fig:youtube_bad} и \ref{fig:yandex_video_bad}), который говорит пользователю о причинах недоступности видеозаписи. Это позволяет провайдерам контента создать более user-friendly интерфейс, однако усложняет задачу детектирования подобных видеозаписей. Задача дополнительно осложняется еще и тем, что текст, говорящий о неработоспособности чаще всего отображается не на странице, содержащей видеозапись, а внутри плеера, что приводит к невозможности использовать подход, основанный на лексических сигнатурах, описанный в вышеупомянутой статье.

\begin{figure}
    \centering
    \includegraphics[width=\textwidth]{images/youtube_bad}
    \caption{Пример неработающего видео на сайте www.youtube.com.}
    \label{fig:youtube_bad}
\end{figure}

\begin{figure}
    \centering
    \includegraphics[width=\textwidth]{images/yandex_video_bad}
    \caption{Пример неработающего видео на сайте видеопоиска.}
    \label{fig:yandex_video_bad}
\end{figure}